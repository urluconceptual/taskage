\chapter{Concluzii}

În concluzie, aplicația Taskage reușeste ceea ce își propune, și anume să reprezinte o alternativă viabilă la toate celelalte aplicații ce permit utilizatorilor doar să-și creeze sarcini de lucru și să le asigneze, automatizând procesul din urmă, pe baza performanțelor trecute ale utilizatorilor, venind cu o soluție de a distribui volumul de muncă într-un mod eficient, fapt ce ușurează munca unui Scrum Master.

În ceea ce privește posibilele direcții de extindere ale acestei aplicații, îmbunătățirile principale ce pot fi facute gravitează în jurul algoritmului de asignare automată pe sarcina de lucru a invidizilor. Principalul gol ce se evidențiază în algoritmul curent este acela că nu ia în considerare frecvența fiecărui tip de însărcinare pentru membrii echipei, astfel neluând în considerare intrări aparent duplicate, cu caracteristici similare. 

În scopul rezolvării acestui punct slab din algoritm putem introduce o nouă metrică ce ia in considerare frecvența de apariție a fiecărui tip de sarcină. O modalitate de a implementa această nouă metrică poate viza calculul proporției tipurilor de task pentru fiecare utilizator în parte, astfel integrând în calculul similarității o modalitate de a inclina sistemul să asigneze o sarcină unui membru al echipei cu mai multă experiență cu tipul respectiv de sarcină.

O altă direcție pentru extinderea aplicației poate fi reprezentată de un mod de a preveni utilizatorii de a intra în burnout, asigurându-se că un membru al echipei nu primește mereu același fel de sarcină, cum ar fi rezolvarea unor defecte. Îmbunătățiri ce ar putea fi aduse aplicației implică atât un mod de a asigna tipuri suficient de variate de sarcini utilizatorilor, cât și adăugarea unui tip de monitorizare al productivității membrilor din echipe, pentru a putea observa și preveni intrarea unuia dintre angajați într-o stare de burnout, judecând după volumul de sarcini completate într-o anumită perioadă de timp.

În final, se poate trage concluzia că o aplicație de gestionare eficientă a sarcinilor de lucru poate fi benefică, în special în contexte în care este aplicată metodologia Agile, pentru a asigura un maxim de performanță și eficiență al angajaților, în scopul maximizării atât al productivității, cât și al fericirii angajaților.