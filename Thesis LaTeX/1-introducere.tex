\chapter{Introducere}

\section{Motivație}

Prin contactul meu de până la momentual acutal cu mediul corporate și, prin extensie, cu metodologia Agile, am observat că o piedică în maximizarea performanței unei echipe o constituie asignarea haotică a sarcinilor de lucru. Aceasta rezultă într-o lipsă de echilibru în volumul de muncă, împreună cu posibilitatea neîncheierii sarcinilor până la data limită a livrării.

Astfel, aplicația centrală acestei lucrări își propune simplificarea gestionării proiectelor din cadrul unei companii și creșterea productivității prin automatizarea procesului de atribuire a sarcinilor de lucru. Utilizatorii cheie sunt managerii, care pot crea panouri cu sarcini de lucru și distribui eficient volumul de muncă, și angajații obișnuiți, care își pot vizualiza sarcinile zilnice și oferi noutăți legat de statusul lor.

\section{Domenii abordate}

Această lucrare are în vedere, în principiu, domeniul dezvoltării aplicațiilor web și al inteligenței arficiale. În cadrul proiectării aplicației, am folosit tehnologii mature precum Java \& Spring pentru back-end, Typescript \& React pentru front-end și baze de date relaționale, prin intermediul PostgreSQL
 [de adaugat detalii dupa implementarea agentului inteligent]

\section{Analiza competitorilor}

Investigând cele mai populare aplicații web de gestiune a sarcinilor de lucru în cadrul unei echipe, competitorii principali identificați sunt: Jira, Monday.com și Azure DevOps.

Fiecare dintre acestea excelează în puncte diferite. Jira este foartre bine adaptat metodologiei Agile, constituind un sistem bun de organizare a unui workflow, Monday.com oferă căi de comunicare pentru echipă, iar Azure DevOps include chiar si funcționalități de version control, facilitând CI/CD. 

Atunci când vine vorba de automatizare, Jira și Monday.com pun la dispoziție un sistem bazat pe triggeri. Utilizatorul trebuie să configureze o regulă de forma ”atunci când condiția … este îndeplinită, execută …”. Astfel, e nevoie de efortul de analiză și decizie al utilizatorului pentru a configura toate aceste condiții. Azure DevOps, pe de alta parte, își axează posibilitățile de automatizare spre sarcinile manuale repetitive, în zona de CI/CD.

\section{Nevoi îndeplinite de aplicație}

Aplicația care este subiectul tezei își propune să construiască un algoritm inteligent care să scadă implicarea utilizatorului în procesul de decizie și analiză a informațiilor adunate, pentru a simplifica munca managerilor sau a Scrum Master-ilor, a crește echitabilitatea împărțirii volumului de muncă, și a crește per total mulțumirea și productivitatea în cadrul echipei.

Mai mult, această procesare a informației poate detecta și nivele neobișnuite de productivitate pentru un anumit angajat, pentru a fi ușor atât pentru acesta, cât și pentru manager, să depisteze semnale de alarmă pentru sănătatea mintală a angajului și să acționeze corespunzător.

\section{Structura lucrării}

[de adaugat descriere a fiecarui capitol dupa finalizarea lucrarii]