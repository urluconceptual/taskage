\chapter{Preliminarii}

\section{Funcționalități principale}
\subsection{Actori}
Principalii actori ai cazurilor de utilizare sunt: 

- Membru de echipă

- Manager

- Administrator

\subsection{Cazuri uzuale de utilizare}
\begin{itemize}
\item Manager:

1. În calitate de manager, aș vrea să adaug sarcini de lucru pentru echipa mea, cu detalii despre criteriile de acceptanță, prioritate, dificultate și termen limită, pentru ca acestea să fie cât mai clare pentru membri echipei.

2. În calitate de manager, aș vrea să pot vizualiza factorii de performanță precum disponibilitatea, volumul de muncă, calificarea, eficiența și atribuțiile fiecărui membru al echipei mele, pentru a evalua corect atribuirea de sarcini de lucru noi.

3. În calitate de manager, aș vrea să pot modifica factorii de performanță precum disponibilitatea, volumul de muncă, calificarea, eficiența și atribuțiile fiecărui membru al echipei mele, pentru ca aceștia să fie mereu actuali.

4. În calitate de manager, aș vrea să atribui automat sarcini de lucru, bazat pe detaliile sarcinii de lucru și pe factorii de performanță angajatului, pentru a le împărți cât mai echitabil și a maximiza productivitatea și șansele ca sarcina să fie îndeplinită până la
termen.

5. În calitate de manager, aș vrea să pot atribui manual sarcini de lucru, bazat pe analiza mea asupra situației, pentru a ajusta eventualele erori ale recomandărilor, a adapta panoul la noi informații relative la disponibilitate, sau a încuraja creșterea/scăderea productivității pentru un anumit membru al echipei

\item Employee:

1. În calitate de angajat obișnuit, aș vrea să îmi vizualizez sarcinile de lucru împreuna cu detalii despre criteriile de acceptanță, prioritate, dificultate și termen limită, pentru a îmi putea organiza activitatea zilnică.

2. În calitate de angajat obișnuit, aș vrea să pot actualiza statusul sarcinilor de lucru pentru ca managerul meu să le cunoască progresul.

3. În calitate de angajat obișnuit, aș vrea să fiu notificat atunci când managerul îmi atribuie o sarcina de lucru nouă, pentru a percepe mai ușor schimbări ale panoului cu sarcinile mele.

\item Administrator:

1. În calitate de administrator, aș vrea să pot crea, actualiza sau elimina utilizatori și echipe, pentru a putea gestiona accesul la platformă și organizarea internă.

2. În calitate de administrator, aș vrea să pot atribui manageri echipelor create, pentru ca fiecare echipă să fie gestionată de un manager.

\end{itemize}


\section{Alegerea tehnologiilor}

\subsection{Typescript\&React}

Pentru partea de front-end a aplicației web am ales să utilizez Typescript ca limbaj și React ca bibliotecă, împreună cu componente reutilizabile din biliboteca de componente UI Ant Design.

Utilizarea limbajului Typescript, în defavoarea limbajului JavaScript, este explicată prin minimizarea erorilor cauzate de inconsistența adusă de faptul ca JavaScript este un limbaj „loosely typed” („dinamically typed”) și de incertitudinea adusă de acest fapt. Astfel, prezența interfețelor și tipurilor de date bine stabilite facilitează dezvoltarea rapidă și stabilă.

În plus, biblioteca React aduce o perspectivă structurată dezvoltării interfeței, prin definirea de componente reutilizabile în cadrul aplicației. O parte din componentele UI sunt utilizate din biblioteca Ant Design, folositoare prin aspectul vizual unitar și formal, acompaniat de o documentație bine pusă la punct. Un dezavantaj al abordării prin componente este comunicarea între acestea, motiv pentru care am folosit state managerul MobX. 

\subsection{Java\&Spring}

Pentru partea de back-end a aplicației web am optat pentru limbajul Java și framework-ul Spring și ecosistemul acestuia, având în vedere opțiunile robuste pentru gestionarea logicii de afaceri, a securității și a interacțiunilor cu baza de date.

Versiunea de Java utilizată este Java 21, pentru a avea acces la cele mai noi update-uri ale limbajului, cu cele mai noi feature-uri, precum clase Record, fire virtuale de execuție, și Sequenced Collections. 

Versiunea de Spring folosită este 6.1.3, ce vine împreună cu Spring Boot 3.2.2, alături de alte dependințe, printre cele mai notabile numărându-se Spring Security și Jakarta Persistence API (JPA).

Cele mai majore utilități puse la dispoziție de Spring și de ecosistemul de dependințe sunt:

- securitatea asigurată prin generarea de JSON Web Tokens pentru autentificarea și autorizarea utilizatorilor și a cererilor HTTP trimise;

- crearea tabelelor și conectarea la baza de date facilitată de ORM-ul JPA;

- injectarea dependințelor prin constructorii claselor, în scopul reducerii codului boilerplate.

[detaliere spring boot, spring security, spring jpa]

\subsection{PostgreSQL}

Pentru persistența datelor am ales PostgreSQL, un sistem open-source de gestiune pentru baze de date, pentru sistemul său robust de asigurare a integrității datelor și abilitatea de a manevra ușor cantități mari de date.